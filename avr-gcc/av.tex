\documentclass[12pt,a4paper]{article}

% Packages
\usepackage{amsmath,amssymb}
\usepackage{geometry}
\geometry{margin=1in}
\usepackage{graphicx}
\usepackage{array}
\usepackage{multicol}
\usepackage{titling} % For custom title positioning

% Remove default headers/footers
\pagestyle{empty}

% Title and Author info
\author{\textbf{Name: Varshini G N} \\ \textbf{ID: COMETFWC031}}
\date{\textbf{Date: August 31, 2025}}
\title{\textbf{GATE 2019 CS Paper Question 30}}

\pretitle{%
  \begin{minipage}{0.2\linewidth}
    \includegraphics[width=2cm]{iiit_logo.png} % replace with logo
  \end{minipage}
  \hfill
  \begin{minipage}{0.7\linewidth}
    \begin{flushright}
}
\posttitle{\end{flushright}\end{minipage}\par\vspace{1cm}}

\begin{document}

\maketitle
\vspace{-1cm} % reduce space below title

\section*{Question}

Consider three 4-variable functions \( f_1, f_2, f_3 \) expressed as sum-of-minterms:

\[
f_1 = \Sigma(0, 2, 5, 8, 14), \quad
f_2 = \Sigma(2, 3, 6, 8, 14, 15), \quad
f_3 = \Sigma(2, 7, 11, 14)
\]

For the circuit with one AND and one XOR gate shown below, express the output function \( f \):

\[
f = (f_1 \cdot f_2) \oplus f_3
\]

\section*{Solution}

\textbf{Step 1: Calculate \(f_1 \cdot f_2\) (AND of minterms)}

\[
f_1 = \{0, 2, 5, 8, 14\}, \quad f_2 = \{2, 3, 6, 8, 14, 15\}
\]

Intersection (AND):

\[
f_1 \cdot f_2 = \{2, 8, 14\}
\]

\textbf{Step 2: XOR with \(f_3 = \{2, 7, 11, 14\}\) yields}

\[
f = \{2, 8, 14\} \oplus \{2, 7, 11, 14\} = \{7, 8, 11\}
\]

\textbf{Therefore:}

\[
f = \Sigma(7, 8, 11)
\]

---

\section*{Truth Table for Inputs \(x_3 x_2 x_1 x_0\) and Output \(f\)}

\begin{tabular}{|c|c|c|c|c|c|}
\hline
\(x_3\) & \(x_2\) & \(x_1\) & \(x_0\) & Decimal & \(f\) \\
\hline
0 & 0 & 0 & 0 & 0 & 0 \\
0 & 0 & 0 & 1 & 1 & 0 \\
0 & 0 & 1 & 0 & 2 & 0 \\
0 & 0 & 1 & 1 & 3 & 0 \\
0 & 1 & 0 & 0 & 4 & 0 \\
0 & 1 & 0 & 1 & 5 & 0 \\
0 & 1 & 1 & 0 & 6 & 0 \\
0 & 1 & 1 & 1 & 7 & 1 \\
1 & 0 & 0 & 0 & 8 & 1 \\
1 & 0 & 0 & 1 & 9 & 0 \\
1 & 0 & 1 & 0 & 10 & 0 \\
1 & 0 & 1 & 1 & 11 & 1 \\
1 & 1 & 0 & 0 & 12 & 0 \\
1 & 1 & 0 & 1 & 13 & 0 \\
1 & 1 & 1 & 0 & 14 & 0 \\
1 & 1 & 1 & 1 & 15 & 0 \\
\hline
\end{tabular}

The output function \( f \) after combining \( f_1, f_2, f_3 \) as given is:

\[
f = \Sigma(7, 8, 11)
\]

Option (A) is correct.

\end{document}
