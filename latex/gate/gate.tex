\documentclass[12pt,a4paper]{article}
\usepackage{amsmath,amssymb}
\usepackage{geometry}
\geometry{margin=1in}
\usepackage{graphicx}
\usepackage{titlesec}
\usepackage{array}

% Remove page numbering
\pagenumbering{gobble}

\begin{document}

\begin{center}
\begin{minipage}{0.20\textwidth}
    \includegraphics[width=\linewidth]{iiit_logo.png}
\end{minipage}
\hfill
\begin{minipage}{0.75\textwidth}
    \centering
    {\Large \textbf{GATE 2018 Question}\\[1ex]}
    {\textbf{Name: Varshini G N}\\[0.5ex]}
    {ID: COMETFWC031}\\[0.5ex]
    \textbf{Date:} \today
\end{minipage}
\end{center}
\vspace{1em}
\hrule
\vspace{1em}

\section*{GATE 2018 CS Q49}

Consider the minterm list form of a Boolean function \( F \) given below.
\[
F(P, Q, R, S) = \sum m(0, 2, 5, 7, 9, 11) + d(3, 8, 10, 12, 14)
\]

Here, \( m \) denotes a minterm and \( d \) denotes a don't care term. The number of essential prime implicants of the function \( F \) is \_\_\_\_\_.

\vspace{1em}

\subsection*{Solution}

\subsubsection*{Truth Table}

\begin{center}
\begin{tabular}{|c|c|c|c|c|c|}
\hline
Decimal & P & Q & R & S & F \\
\hline
0  & 0 & 0 & 0 & 0 & 1   \\
1  & 0 & 0 & 0 & 1 & 0   \\
2  & 0 & 0 & 1 & 0 & 1   \\
3  & 0 & 0 & 1 & 1 & X   \\
4  & 0 & 1 & 0 & 0 & 0   \\
5  & 0 & 1 & 0 & 1 & 1   \\
6  & 0 & 1 & 1 & 0 & 0   \\
7  & 0 & 1 & 1 & 1 & 1   \\
8  & 1 & 0 & 0 & 0 & X   \\
9  & 1 & 0 & 0 & 1 & 1   \\
10 & 1 & 0 & 1 & 0 & X   \\
11 & 1 & 0 & 1 & 1 & 1   \\
12 & 1 & 1 & 0 & 0 & X   \\
13 & 1 & 1 & 0 & 1 & 0   \\
14 & 1 & 1 & 1 & 0 & X   \\
15 & 1 & 1 & 1 & 1 & 0   \\
\hline
\end{tabular}
\end{center}

\vspace{1ex}

\subsubsection*{K-Map Representation (4 Variables)}

Below is one possible K-map arrangement (\(PQ\) as rows, \(RS\) as columns):

\[
\begin{array}{c|cccc}
PQ\backslash RS & 00 & 01 & 11 & 10 \\
\hline
00 & 1  & 0 & X & 1 \\
01 & 0  & 1 & 1 & 0 \\
11 & X  & 0 & 0 & X \\
10 & X  & 1 & 1 & X \\
\end{array}
\]

\textbf{Legend:} ‘1’ = Minterm, ‘X’ = Don't care, ‘0’ = Not included.

\vspace{1ex}

\subsubsection*{Essential Prime Implicant Calculation}

Prime implicant groups:
\begin{itemize}
    \item Group 1: Covers minterms 0, 2 (essential)
    \item Group 2: Covers minterms 5, 7 (essential)
    \item Group 3: Covers minterms 9, 11 (essential)
\end{itemize}

So, the number of essential prime implicants is:
\[
\boxed{3}
\]

\textbf{Final Answer:} There are \textbf{3 essential prime implicants}.

\end{document}