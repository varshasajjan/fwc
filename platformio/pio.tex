\documentclass[12pt,a4paper]{article}
\usepackage{geometry}
\geometry{margin=1in}
\usepackage{graphicx}
\usepackage{amsmath}
\usepackage{enumitem}
\usepackage{xcolor}
\usepackage{titling} % lets us customize \maketitle

% ---- customize these ----
\title{\textbf{GATE CS Paper 2019 Question 4}}
\author{Varshini G N \\ ID: COMETFWC031}
\date{August 29, 2025}
\newcommand{\LogoFile}{iiit_logo.png} % put your logo file in same folder
% -------------------------

\pretitle{%
  \begin{minipage}[t]{0.48\textwidth}
    \includegraphics[height=1.8cm]{\LogoFile}
  \end{minipage}
  \begin{minipage}[t]{0.48\textwidth}\raggedleft
}
\posttitle{\end{minipage}\par\vspace{0.5em}\hrule\vspace{1em}}

\begin{document}

\maketitle

\section*{Question}
In 16-bit 2's complement representation, the decimal number $-28$ is:

\begin{enumerate}[label=(\Alph*)]
  \item 1111 1111 0001 1100
  \item 0000 0000 1110 0100
  \item 1111 1111 1110 0100
  \item 1000 0000 1110 0100
\end{enumerate}

\section*{Solution}
\textbf{Step 1:} Write $+28$ in 16-bit binary:
\[
28 = 0000\ 0000\ 0001\ 1100
\]

\textbf{Step 2:} Take the 1's complement (invert all bits):
\[
1111\ 1111\ 1110\ 0011
\]

\textbf{Step 3:} Add 1 to obtain the 2's complement:
\[
1111\ 1111\ 1110\ 0100
\]

Therefore, the correct answer is
\[
\boxed{\textcolor{red}{\text{(C) }1111\ 1111\ 1110\ 0100}}.
\]

\end{document}
